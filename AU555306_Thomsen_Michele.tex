\documentclass{article}
\usepackage[utf8]{inputenc}

\title{Digital Methods: Learning Journal Template}
\author{Michele Høg Plet Thomsen}
\date{Autumn 2019}

\begin{document}

\maketitle

\section{Today's Date}
\subsection{Thoughts / Intentions}
\subsection{Homework or Project}
\subsection{Results}
\subsection{Final Thoughts}

\pagebreak{}

\section{31/10/2019}
\subsection{Thoughts / Intentions}

\textbf{11:27am}: Anne Sofie and I went to the library to make the homework for next week. I am currently worried about whether I find the homework too difficult, but I am fairly optimistic after the hands on class. I do, however, have difficulties with figureing out where to begin. 


\subsection{Homework}
\textbf{Assignment nr. 1: }
\item We began with copying the data for StopwordListVoyant into \end{itemize} Regular expression, so that it can be used for R \end{itemize}
\begin{itemize}
\item  The data was listed under each other by using the code ([a-z,æ,ø,å,é].+) square bracket reverse / n square bracket, however, we needed them to be listed side by side. This caused huge problems, in figuring out how to do this. We found out that the code (square brackets reverse / n square brackets) was correct, but our placement was wrong. 
\item  We still had some numbers that needed to be removed, which we did by using the code (reverse / d)square bracket n square bracket and then "dollar sign one"
\end{itemize}
\textbf{Assignment nr. 2:}
For this assignement, we took the data from StopwordListForR and put it into Regular Expression, and then we need to convert the data, so that it can be used for Voyant.
\begin{itemize}
\item  We began removing our " and , by using the codes (,) and (").
\item  Then we used the code ( reverse / s)
\end{itemize}
In order to line our data under each other, we used the code dollar sign one reverse / n .
\end{itemize}

\subsection{Results}
Assignment 1: 
\begin{itemize}
\item Link to result: https://regex101.com/r/QaQLi3/1?fbclid=IwAR2ByCMGYpKVGxS8ApzMSuiFnmuja3dbbpUcZMXZo21rlpVyw-dUIbaKnYg 
\end{itemize}
Assignment 2:
\begin{itemize}
\item Link to result: https://regex101.com/r/DJNnMS/1?fbclid=IwAR1PoVTik72s3O8PHcwlSt-9FqlYFtMVyqe5CRd932uHlte895EZdnDtVNc 
\subsection{Final Thoughts}

\textbf{1:11pm}: It was frustrating at first. The codes were difficult to find and i think we thought about it too much. The second assignment went much smoother, because we chose to take it one step at a time, instead of doing it all in once. 


\section{07/11/2019}

\subsection{11.50am I came up with an idea for the exam}

For the exam I would like to analyse the different minutes to see if this gives an idea on how the widows position was in the society. However, when I searched for their data sheet in Kvinfo.dk, it was very messy. I then came with the idea to make a nice data sheet on the minutes where widows was mentioned. This, however, seems very difficult. I have no idea where to start, which tools to use and etc. So I contacted Max. 

\subsection{Thoughts and intentions}

\textbf{11.50am - I came up with an idea for the exam}

For the exam I would like to analyse the different minutes to see if this gives an idea on how the widows position was in the society. However, when I searched for their data sheet in Kvinfo.dk, it was very messy. I then came with the idea to make a nice data sheet on the minutes where widows was mentioned. This, however, seems very difficult. I have no idea where to start, which tools to use and etc. So I contacted Max. 

\subsection{Project}
\textbf{11.53am Max' office}

Max told me that this idea was possible by using R and html coding. It would take some work to do, however, he is sure that the project would have a good outcome, if I put in the time and effort. I still had my concerns whether I have the skills for it, but he ensured me that he would help me during the process. He also recommended me to read Xpath tutorial from W3schools.com. My next step now, is to read the tutorial and cross my fingers and hope I understands it. 

\textbf{2pm Adela's office}

I pitched my idea to Adela and she gave me the green light. It was ambitious, but with max' help I should be doing just fine. I did, however, need to be observant on the ethics of my paper and I need to take it seriously, since I am using Webscraping and if I am doing it wrong, it could cause legal problems

\subsection{Results}
I think I have my final project.
I want to solve the problematics regarding data collecting from Kvinfo.dk, so that it is more approachable for other scholars, who would like to use their data in their projects.
When doing this I will be using R, webscrabing, Regular Expressions and html coding. 

\subsection{Final Thoughts}
I think, if I am doing this right, I have a pretty good project


\section{22/11/2019}

\subsection{Thoughts and Intentions}
I would say that R is fairly comprehensible to understand and work with. Many of the codes make sense and the data carpentry is very helpful. When that is said, it takes time for me to learn the codes and remembering them (because there are so so many of them)

\subsection{Homework for week 5}

I tried doing the homework by the tools and methods we learned in class. However, when i checked the solutions, I found out that I hadn't done it the way the solutions had done it, I think i missed a step, but otherwise it was the same result. 

I am having great difficulties with uploading an image in overleaf. I have tried google, I have tried youtube videos, however, nothing has worked so far. I think it has something to do with my file being a png file instead of pkg or jpg, but I cannot seem to edit this on my computer. It simply will not allow me. 

\begin{figure}[htp]
    \centering
    \includegraphics[width=4cm]{R-lektie.png}
    \usepackage{graphicx}
    \caption{Homework for week 5}
    \label{R-scipt figure}
\end{figure}
I clearly can't upload any images, but I tried doing the homework by the tools and methods we learned by class. 

\subsection{Result}
The task was to change the values in "membassoc" in the SAFI-data, by replacing the data "N/A" with "undetermined"
I did not manage to change my values to "undetermined", but it did change to "NULL". 

\subsection{Final thought}
This is actually quite fun when you first get the (sort of/almost) hang of it. 


\section{26/11/2019}
\subsection{Thoughts and Intentions}
I am really looking forward to the workshop on webscraping with Max, since I have found it difficult to begin with my project before learning som more.

\subsection{Project}
\textbf{Webscraping with Max}
I learned how to webscrape by using html codes, xpath and R. My webside Kvinfo, where I will webscrabe from, is not, however, based on html coding. Instead it is a Javascript based webside. This mean that I have to use Gitbash for my project. 

By using Gitbash, I downloaded all the all the records on enke (widow) on kvinfo.dk. This gave me almost 10000 links of records. 
However, not all of those records have any context. In fact, thousands of those are empty records, which means I have to clean it up. This will again be done by using Gitbash. 

\subsection{Results}
When I got home, I managed to successfully download all the records from kvinfo.dk, however, the code that I should use for deleting all the empty records, does not work. I know, since this was also the problem during the workshop, that a -t needs to be added somewhere in the code, however, I cannot remember where. I have therefore sent an email to Max and is currently waiting for a reply.

\subsection{Final Thoughts}
I hope I succeeds with this project and won't come accross too many problems that I cannot solve myself. 

\section{28/11/2019}
\subsection{Thoughts and intentions}
I really hope that I can do the homework at hope as well. I am having difficulties with remembering what to do, when I am on my own and not in class. I think that I might overthink my steps. 

\subsection{Homework for week 6}
I think it went really good this time. 

For the first assignment I used the correct values, however the wrong coding. I used "group by" and "summarise" instead of count.

When doing the second assignment, I had no clue on what to do, so i Used the solution. This is also the case with assignment nr. 3. 

\subsection{Results}
Assignment nr. 1: 2 meals = 52 households, 3 meals = 79 households

Assignment nr. 2: mean number of households = (Chirodzo) 7,08                       (God) 6,86 (Ruaca) 7,57
                  min = (Chirodzo) 2 (God) 3 (Ruaca) 2
                  max = (Chirodzo) 12 (God) 15 (Ruaca) 19
                  
Assignment nr. 3: November 2016 = 19 members of a household
                  December 2016 = 12 members
                  April 2017 = 17 members
                  May 2017 = 15 members
                  June 2017 = 15 members

\subsection{Final thoughts}
This was not as painfull as I have dreaded it would be.


\end{document}
