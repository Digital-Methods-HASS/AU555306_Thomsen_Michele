\documentclass{article}
\usepackage[utf8]{inputenc}

\title{Digital Methods: Learning Journal Template}
\author{Michele Høg Plet Thomsen}
\date{Autumn 2019}

\begin{document}

\maketitle

\section{Today's Date}
\subsection{Thoughts / Intentions}
\subsection{Action}
\subsection{Results}
\subsection{Final Thoughts}

\pagebreak{}

\section{31/10/2019}
\subsection{Thoughts / Intentions}

\textbf{11:27am}: Anne Sofie and I went to the library to make the homework for next week. I am currently worried about whether I find the homework too difficult, but I am fairly optimistic after the hands on class. I do, however, have difficulties with figureing out where to begin. 


\subsection{Action}
\textbf{Assignment nr. 1: }
\item We began with copying the data for StopwordListVoyant into \end{itemize} Regular expression, so that it can be used for R \end{itemize}
\begin{itemize}
\item  The data was listed under each other by using the code ([a-z,æ,ø,å,é].+) square bracket reverse / n square bracket, however, we needed them to be listed side by side. This caused huge problems, in figuring out how to do this. We found out that the code (square brackets reverse / n square brackets) was correct, but our placement was wrong. 
\item  We still had some numbers that needed to be removed, which we did by using the code (reverse / d)square bracket n square bracket and then "dollar sign one"
\end{itemize}
\textbf{Assignment nr. 2:}
For this assignement, we took the data from StopwordListForR and put it into Regular Expression, and then we need to convert the data, so that it can be used for Voyant.
\begin{itemize}
\item  We began removing our " and , by using the codes (,) and (").
\item  Then we used the code ( reverse / s)
\end{itemize}
In order to line our data under each other, we used the code dollar sign one reverse / n .
\end{itemize}

\subsection{Results}
Assignment 1: 
\begin{itemize}
\item Link to result: https://regex101.com/r/QaQLi3/1?fbclid=IwAR2ByCMGYpKVGxS8ApzMSuiFnmuja3dbbpUcZMXZo21rlpVyw-dUIbaKnYg 
\end{itemize}
Assignment 2:
\begin{itemize}
\item Link to result: https://regex101.com/r/DJNnMS/1?fbclid=IwAR1PoVTik72s3O8PHcwlSt-9FqlYFtMVyqe5CRd932uHlte895EZdnDtVNc 
\subsection{Final Thoughts}

\textbf{1:11pm}: It was frustrating at first. The codes were difficult to find and i think we thought about it too much. The second assignment went much smoother, because we chose to take it one step at a time, instead of doing it all in once. 


\section{07/11/2019}

\subsection{11.50am I came up with an idea for the exam}

For the exam I would like to analyse the different minutes to see if this gives an idea on how the widows position was in the society. However, when I searched for their data sheet in Kvinfo.dk, it was very messy. I then came with the idea to make a nice data sheet on the minutes where widows was mentioned. This, however, seems very difficult. I have no idea where to start, which tools to use and etc. So I contacted Max. 

\subsection{11.53am Max' office}

Max told me that this idea was possible by using R and html coding. It would take some work to do, however, he is sure that the project would have a good outcome, if I put in the time and effort. I still had my concerns whether I have the skills for it, but he ensured me that he would help me during the process. He also recommended me to read Xpath tutorial from W3schools.com. My next step now, is to read the tutorial and cross my fingers and hope I understands it. 

\section{22/11/2019}

\subsection{10.11am Homework for week 5}

I am having great difficulties with uploading an image in overleaf. I have tried google, I have tried youtube videos, however, nothing has worked so far. I think it has something to do with my file being a png file instead of pkg or jpg, but I cannot seem to edit this on my computer. It simply will not allow me. 

\begin{figure}[htp]
    \centering
    \includegraphics[width=4cm]{R-lektie.png}
    \usepackage{graphicx}
    \caption{Homework for week 5}
    \label{R-scipt figure}
\end{figure}
I clearly couldn't upload any images, but I tried doing the homework by the tools and methods we learned by class. However, when i checked the solutions, I found out that I hadn't done it the way the solutions had done it, I think i missed a step, but otherwise it was the same result. 
\subsection{Action}

Scoping Exercise
\begin{itemize}
\item Open scoping exercise in cloudstor
\item Open blank document in overleaf
\item Written introduction under intro section then created a new section
\item Recompiled document 
\item Opened “A day in the life worksheet”
\item Finished jobs section and created new section using pains
\item Repeated section command for pain relievers
\item ERROR in Overleaf - unable to make the first point under the section in line with the reset of the points.
\item Attempted more line spaces and finding a command in “\” to correct however unsuccessful - will ask in class
\item Completed document and saved as PDF
\item Uploading to cloudstor Sophie Wallace Scoping Exercise folder.
\item Made new repository in github under scoping-exercise 
\item Added description and committed with a README
\end{itemize}


Overleaf —> Github

\begin{itemize}
\item Found github sync command in left Menu tab on overleaf
\item Clicked authorise and was met with “This project is not linked to a GitHub repository. You can create a repository for it in GitHub”
\item Now following button to ‘create github repository’
\item Opens export project to github and asks to create new repository with description
\item Created and clicked ‘public’ option
\item \textbf{ERROR}: repository creation failed “please check that the repository name is valid,  and that you have permission to create the repository.
\item \textbf{FIXED}: Changed the name scopingexercise and worked - error may have been in creating a repository with the exact same name as a previous one.
\item I now have two repositories with scoping exercises
\end{itemize}

\subsection{Final Thoughts}

\textbf{11:49am}

\begin{itemize}
\item Scoping exercise: I found this exercise to be relevant and thought-provoking in how I can utilize this unit in making my thesis easier to manage and organize. It was straight forward and interesting. 
\item Overleaf: I’m glad Overleaf worked for me this time and I was able to successfully create a document. I would still love to learn how to properly format and become comfortable with utilizing its features. 
\item Github: Although I am able to upload and create repositories, I don’t think I have a full grasp of its features or process. I also do not think that I am connected to a shared FOAR705 github so I will need to figure that out in class.
\item Cloudstor: very straightforward and easy to use.
\end{itemize}

\section{20/08/2019}
\subsection{Thoughts/Intentions}
\textbf{7:40pm}:  I've set up another monitor on my desk to help with the multiple tabs that need to opened at once and feeling more organized. Planning to work through data carpentry.

\textbf{9:05pm}: I want to quickly try again at the data sheet after reading other students processes for this task in their learning journals. 

\subsection{Action}

Dates as Data 
\begin{itemize}
\item Open 'Dates as Data' section in 'Data Organization in Spreadsheets for Social Sciences'
\item Downloaded SAFI dated spreadsheet
\item Had to google "how to extract components of date into new columns" because I was unsure what this meant (novice excel user here)
\item Created new columns adjacent to interview dates and named each column "day" , "month", "year" 
\item Wrote in Day column next to interview date 17/11/2016 =DAY(17) 
\item \textbf{ERROR}: the cell automated to 17/01/1900
\item Wrote in the month column =MONTH(11) - ERROR: automated to 01/01/1900
\item Retyped under Day column =Day(B2) to refer to the date - \textbf{ERROR}: There are one or more circular references where a formula refers to its own cell either directly or indirectly. This might cause them to calculate incorrectly.
\item Re-downloading spreadsheet and starting fresh.
\item \textbf{ERROR} - same issue where =day(17) automates to 17/01/1900
\item Filling it all out with the automated system, I see that the excels date systems (1990) is meant to be relevant, I'm confused to how it applies but will ask.
\item Just noticed the solution tab on Data Carpentry and going to enter dates manually without an automated function and ask question in class.
\item Added formula to each column underneath i.e. day column has =day(b1:b15)
\item #VALUE! appears in the formula cell. 
\item Leaving for now.
\end{itemize}

Dates as Data (Attempt 2)

\begin{itemize}
\item Opened new spreadsheet 
\item Remembered to create a new tab rather than to manipulate the raw data
\item Copied interview dates into column A in new tab
\item Named column B - Day, column C - month and column D - year
\item Entered the formula =day(A2) into B2 -\textbf{ SUCCESS}
\item Clicked and dragged the bottom right corner of cell B2 down entire column
\item Day column has now extracted the dates correctly.
\item Copying process with month and year column -\textbf{SUCCESS}
\item Adding 17/11 in cell A16, automates to 17-Nov in interview date section and includes the year 2019 in the column
\item If no year is specified the current year must be inserted.
\item End of dates as data exercise.

\end{itemize}





\subsection{Final Thoughts}
\textbf{8:30pm} Very frustrating that what was assumed as a simple task has taken a lot of back and forth and time but hoping once I'm able to reiterate these errors with someone it will be clear and straight forward. 


\textbf{9:22pm} After reading through other learning journals and it became a lot clearer the mistakes I was making. I wish there were clearer instructions in these sites to assist with novice users such as myself. I am glad that I was able to successfully complete the exercise after seeing similar steps. This task reminded me that I must use a new tab when adding and working with data in excel. 



\end{document}
